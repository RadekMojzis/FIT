%2/2 monografie (alespoň jedna musí být cizojazyčná),
%3/3 elektronické dokumenty (online),
%1/1 seriálovou publikaci (tištěný časopis či sborník z konference),
%2/2 články v seriálové publikaci (pokud možno zahraniční tištěné články),
%2/2 kvalifikační práce (bakalářské, diplomové nebo disertační).
\documentclass[11pt,a4paper,titlepage]{article}
\usepackage[left=2cm,text={17cm, 24cm},top=3cm]{geometry}
\usepackage{times}
\usepackage[czech]{babel}
\usepackage[utf8]{inputenc}
\bibliographystyle{czplain}
\usepackage{etoolbox}
\apptocmd{\thebibliography}{\raggedright}{}{}
%\usepackage{hyperref}
%\usepackage[hyphenbreaks]{breakurl}

\providecommand{\czq}[1]{\quotedblbase #1\textquotedblleft}

\begin{document}
%%%%%%%%% ----- Headder
\thispagestyle{empty}
\begin{center}

\textsc{
\Huge{Vysoké učení technické v~Brně \\}
\huge{Fakulta informačních technologií\\}
}
\vspace{\stretch{0.382}}
 

\LARGE{Typografie a publikování, 4. projekt. \\}
\Huge{Bibliografické citace\\}
\vspace{\stretch{0.618}}

\end{center}

\Large{\today} \hfill Radek Mojžíš

%%%%%%%%%%% ---- Actual document
\newpage
\setcounter{page}{1}
\newpage
\section{\LaTeX}
\subsection{Co vlastně vůbec \LaTeX je?}
\LaTeX je obsáhlá sada značkovacích příkazů, používaná s výkoným
sázecím programem \TeX pro přípravu široké škály dokumentů,
od vědeckých článků až po komplexní knihy.

\LaTeX, stejně jako \TeX je otevřený softwarový projekt, dostupný zdarma.
Jeho jádro je udržováno projektovou skupinou \LaTeX3 ale také využívá rozšíření 
napsaná stovkami uživatelů/přispěvatelů a~to~i~se~všemi výhodami i~nevýhodami
takové demokracie.
\cite{book:guide-to-latex}

\subsection{Kdy \LaTeX použít?}
LaTeX, popř. TeX, se dá využít na~psaní téměř libovolného dokumentu.
Od obyčejných stránkových referátů až po rozsáhlé seminární práce. 
Výhody psaní dokumentů v LaTeXu ale pocítí hlavně ti, kteří píší
nějakou vědeckou práci, kam musí vkládat zdrojové kódy, matematické
rovnice atd.
\cite{web:kdy-pouzit-latex}

\section{Typografie}
\subsection{časté chyby}
Teď, když víme, co máme k~sázení dokumentů používat, je vhodné zmínit
časté typografické chyby.

Tečky, čárky, vykřičníky, otazníky neboli interpunkční znaménka. 
Za všemi se dělá mezera. Před nimi nikoli.
Jedinou výjimkou jsou tři tečky neboli výpustka. Můžeme je psát jako 3 tečky 
\czq{...} nebo jako znak výpustka \czq{…}. Není v tom takový rozdíl. 
\cite{web:chyby}

Už nesčetněkrát mi někdo poslal podklady pro ceník,
leták nebo brožuru a občas jsem narazil na to, že je cena špatně napsaná.
Pokud totiž napíšete 1~599,-~Kč, je to špatně.

Jak je to tedy správně? Jak správně typograficky
zapíšeme cenu? Nejjednodušší způsob je 1~599~Kč, 
sám to tak nejčastěji píšu. Pokud je z kontextu 
jasné, o~jakou měnu se jedná, lze použít 1 599,-. 
Existují ale i~další správné způsoby zápisu.
\cite{web:zahadna-typografie-3-nejcastejsi-typograficke-chyby}

U nejméně pěticiferných čísel se číslo člení do skupin po třech číslicích,
a to na obou stranách od desetinné čárky, přičemž se mezi nimi píše nedělitelná
mezera. U čtyřciferných čísel je situace složitější: letopočty se nedělí nikdy,
ostatní čtyřciferná čísla se podle mnoha typografických příruček mají také dělit (např. \czq{7 530})
\cite{book:prakticka-typografie}

\section{Procesory}
\subsection{Jak navrhnout procesor?}
Navrhnout procesor je složitý úkol, z pravidla se to v dnešní době
dělá pomocí jazyka VHDL. Procesor se skládá z registrů, aritmeticko-logické jednotky,
modulu pro počítání v plovoucí desetinné čárce, řadiče a dalších komponent. \cite{bc:navrh-procesoru}

\subsection{Zřetězení, více jader?}
Procesory dovedou provádět více instrukcí najednou, a to nejen díky tomu, že mají více jader.
Při zřetězeném spracování instrukcí se vykonává několik instrukcí najednou na jednom jádře.
Jsou zde ovšem jisté hazardy, například datový, řídící nebo strukturální. \cite{bc:navrh-procesoru-CODAL}

\subsection{AMD Ryzen}
AMD se dostává z pozice druhořadého výrobce procesorů. Po době
která se zdála už věčností, po roce očekávání je Ryzen konečně zde
aby zatřásl s procesorovým stasuem quo.
Výkon je lepší, než mnozí očekávali a Intel už začíná mluvit o více jádrech
a větší flexibilitě v nadcházejících produktech ve své řadě procesorů.
Milovníci AMD pochopitelně, již pomalu otevírají šampaňské aby oslavovali.
Konkurence je tady.
\cite{article:amd-ryzen}
\newpage
\bibliography{xmojzi07}
\end{document}
